\nonstopmode{}
\documentclass[a4paper]{book}
\usepackage[times,inconsolata,hyper]{Rd}
\usepackage{makeidx}
\usepackage[utf8,latin1]{inputenc}
% \usepackage{graphicx} % @USE GRAPHICX@
\makeindex{}
\begin{document}
\chapter*{}
\begin{center}
{\textbf{\huge Package `iForm'}}
\par\bigskip{\large \today}
\end{center}
\begin{description}
\raggedright{}
\item[Type]\AsIs{Package}
\item[Title]\AsIs{Forward Selection Under Marginality Principle}
\item[Version]\AsIs{1.0}
\item[Date]\AsIs{2016-03-29}
\item[Author]\AsIs{Kirk Gosik }\email{kgosik@broadinstitute.org}\AsIs{}
\item[Maintainer]\AsIs{Kirk Gosik }\email{kgosik@broadinstitute.org}\AsIs{}
\item[Description]\AsIs{Extended variable selection approaches to jointly model main and
interaction effects from high-dimensional data.}
\item[License]\AsIs{GPL-3}
\item[LazyData]\AsIs{TRUE}
\item[RoxygenNote]\AsIs{6.0.1.9000}
\item[Suggests]\AsIs{knitr, rmarkdown}
\item[VignetteBuilder]\AsIs{knitr}
\end{description}
\Rdcontents{\R{} topics documented:}
\inputencoding{utf8}
\HeaderA{iForm}{Interaction Screening for Ultra-High Dimensional Data}{iForm}
%
\begin{Description}\relax
Extended variable selection approaches to jointly model main and interaction effects from high-dimensional data orignally proposed by Hao and Zhang (2014) and extended by Gosik and Wu (2016).
Based on a greedy forward approach, their model can identify all possible interaction effects through two algorithms, iFORT and iFORM, which have been proved to possess sure screening property in an ultrahigh-dimensional setting.
\end{Description}
%
\begin{Usage}
\begin{verbatim}
iForm(formula, data, strong = TRUE, higher_order = FALSE)
\end{verbatim}
\end{Usage}
%
\begin{Arguments}
\begin{ldescription}
\item[\code{formula}] an object of class "formula" (or one that can be coerced to that class): a symbolic description of the model to be fitted. The details of model specification are given under ‘Details’.

\item[\code{data}] data.frame of your data with the response and all p predictors

\item[\code{strong}] logical TRUE to use strong heredity or FALSE to use weak heredity (default TRUE)

\item[\code{higher\_order}] logical TRUE indicating to include order-3 interactions in the search (default FALSE)
\end{ldescription}
\end{Arguments}
%
\begin{Details}\relax
Runs the iForm selection procedure on the dataset and returns a linear model
of the final selected model.
\end{Details}
%
\begin{Value}
a summary of the linear model returned after the selection procedure
\end{Value}
%
\begin{Author}\relax
Kirk Gosik
\end{Author}
\inputencoding{utf8}
\HeaderA{iformselect}{Inner workings for different selections under different higher-orders and strength of marginality.}{iformselect}
%
\begin{Description}\relax
Extended variable selection approaches to jointly model main and interaction effects from high-dimensional data orignally proposed by Hao and Zhang (2014) and extended by Gosik and Wu (2016).
Based on a greedy forward approach, their model can identify all possible interaction effects through two algorithms, iFORT and iFORM, which have been proved to possess sure screening property in an ultrahigh-dimensional setting.
\end{Description}
%
\begin{Usage}
\begin{verbatim}
iformselect(x, y, p, n, candidate, solution, model, bic, step, strong,
  higher_order)
\end{verbatim}
\end{Usage}
%
\begin{Arguments}
\begin{ldescription}
\item[\code{x}] design matrix of predictors

\item[\code{y}] response variable

\item[\code{p}] the size of the predictor set

\item[\code{n}] the size of the number of observations

\item[\code{candidate}] the current candidate set of predictors to select from

\item[\code{solution}] the current set of predictors already selected

\item[\code{model}] the set of predictors to use in the final model

\item[\code{bic}] the cutoff value for determining the model set

\item[\code{step}] the step in the iteration currently on

\item[\code{strong}] indicator of the strength of marginality to be used

\item[\code{higher\_order}] logical TRUE indicating to include order-3 interactions in the search (default FALSE)
\end{ldescription}
\end{Arguments}
%
\begin{Details}\relax
Runs the iForm selection procedure on the dataset and returns a linear model
of the final selected model.
\end{Details}
%
\begin{Value}
a summary of the linear model returned after the selection procedure
\end{Value}
%
\begin{Author}\relax
Kirk Gosik
\end{Author}
\printindex{}
\end{document}
