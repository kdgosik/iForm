\nonstopmode{}
\documentclass[letterpaper]{book}
\usepackage[times,inconsolata,hyper]{Rd}
\usepackage{makeidx}
\usepackage[utf8,latin1]{inputenc}
% \usepackage{graphicx} % @USE GRAPHICX@
\makeindex{}
\begin{document}
\chapter*{}
\begin{center}
{\textbf{\huge Package `iForm'}}
\par\bigskip{\large \today}
\end{center}
\begin{description}
\raggedright{}
\item[Type]\AsIs{Package}
\item[Title]\AsIs{Forward Selection Under Marginality Principle}
\item[Version]\AsIs{1.0}
\item[Date]\AsIs{2016-03-29}
\item[Author]\AsIs{Kirk Gosik }\email{kgosik@broadinstitute.org}\AsIs{}
\item[Maintainer]\AsIs{Kirk Gosik }\email{kgosik@broadinstitute.org}\AsIs{}
\item[Description]\AsIs{Extended variable selection approaches to jointly model main and
interaction effects from high-dimensional data.}
\item[License]\AsIs{GPL-3}
\item[LazyData]\AsIs{TRUE}
\item[RoxygenNote]\AsIs{6.0.1.9000}
\item[Suggests]\AsIs{knitr,rmarkdown}
\item[VignetteBuilder]\AsIs{knitr}
\item[NeedsCompilation]\AsIs{no}
\end{description}
\Rdcontents{\R{} topics documented:}
\inputencoding{utf8}
\HeaderA{iForm}{Interaction Screening for Ultra-High Dimensional Data}{iForm}
%
\begin{Description}\relax
Extended variable selection approaches to jointly model main and interaction effects from high-dimensional data orignally proposed by Hao and Zhang (2014) and extended by Gosik and Wu (2016).
Based on a greedy forward approach, their model can identify all possible interaction effects through two algorithms, iFORT and iFORM, which have been proved to possess sure screening property in an ultrahigh-dimensional setting.
\end{Description}
%
\begin{Usage}
\begin{verbatim}
iForm(formula, data, heredity = "strong", higher_order = FALSE)
\end{verbatim}
\end{Usage}
%
\begin{Arguments}
\begin{ldescription}
\item[\code{formula}] an object of class "formula" (or one that can be coerced to that class): a symbolic description of the model to be fitted. The details of model specification are given under ‘Details’.

\item[\code{data}] data.frame of your data with the response and all p predictors

\item[\code{heredity}] a string specifying the heredity to be considered. NULL, weak, strong

\item[\code{higher\_order}] logical TRUE indicating to include order-3 interactions in the search (default FALSE)
\end{ldescription}
\end{Arguments}
%
\begin{Details}\relax
Runs the iFORM selection procedure on the dataset and returns a linear model
of the final selected model.  The model is of an R object of class "lm"
\end{Details}
%
\begin{Value}
a summary of the linear model returned after the selection procedure
\end{Value}
%
\begin{Author}\relax
Kirk Gosik
\end{Author}
%
\begin{SeeAlso}\relax
\code{lm}

\code{model.frame}
\end{SeeAlso}
%
\begin{Examples}
\begin{ExampleCode}
iForm(formula = hp ~ ., data = mtcars, heredity = "strong", higher_order = FALSE)
\end{ExampleCode}
\end{Examples}
\inputencoding{utf8}
\HeaderA{iformselect}{Selection for iForm procedure}{iformselect}
%
\begin{Description}\relax
This is a helper function to run the selection procedure under differen heredity principles
and different levels of interactions included in the selection.
\end{Description}
%
\begin{Usage}
\begin{verbatim}
iformselect(x, y, p, n, C, S, bic, heredity, higher_order)
\end{verbatim}
\end{Usage}
%
\begin{Arguments}
\begin{ldescription}
\item[\code{x}] data.frame of your data with all p predictors

\item[\code{y}] vector of observed responses

\item[\code{p}] number of predictors in the dataset

\item[\code{n}] number of observations in the dataset

\item[\code{C}] vector of candidate predictors to consider in this step of the procedure

\item[\code{S}] vector of solution predictor selected from previous steps of the procedure

\item[\code{bic}] vector of bic values calculated for each step of the procedure

\item[\code{heredity}] a string specifying the heredity to be considered. NULL, weak, strong

\item[\code{higher\_order}] logical TRUE indicating to include order-3 interactions in the search (default FALSE)
\end{ldescription}
\end{Arguments}
%
\begin{Details}\relax
Runs the iFORM selection procedure for specifed heredity and level of interactions.
It returns the solution to be fit from \code{iForm}
\end{Details}
%
\begin{Value}
the response vector, the solution set of predictors and the calculated bic values
\end{Value}
%
\begin{Author}\relax
Kirk Gosik
\end{Author}
\inputencoding{utf8}
\HeaderA{rss\_map\_func}{Finding minimum RSS}{rss.Rul.map.Rul.func}
%
\begin{Description}\relax
Helper function to take in the candidate set and solution set along with the observations
and previous data to calculate the residual sum of sqaures for each of the candidate
predictors given what has already been selected.
\end{Description}
%
\begin{Usage}
\begin{verbatim}
rss_map_func(C, S, y, data)
\end{verbatim}
\end{Usage}
%
\begin{Arguments}
\begin{ldescription}
\item[\code{C}] vector of candidate predictors to consider in this step of the procedure

\item[\code{S}] vector of solution predictor selected from previous steps of the procedure

\item[\code{y}] vector of observed responses

\item[\code{data}] data.frame of your data with the response and all p predictors
\end{ldescription}
\end{Arguments}
%
\begin{Details}\relax
Mapping function to calculcate the residual sum of squares for each of the candidate predictors
\end{Details}
%
\begin{Value}
A vector of the RSS values for each candidate predictor
\end{Value}
%
\begin{Author}\relax
Kirk Gosik
\end{Author}
\inputencoding{utf8}
\HeaderA{strong\_order2}{Creating interactions based off of strong heredity principle}{strong.Rul.order2}
%
\begin{Description}\relax
Helper function to give all possible order-2 interactions following the strong
heredity principle.
\end{Description}
%
\begin{Usage}
\begin{verbatim}
strong_order2(S, data)
\end{verbatim}
\end{Usage}
%
\begin{Arguments}
\begin{ldescription}
\item[\code{S}] vector of solution predictor selected from previous steps of the procedure

\item[\code{data}] data.frame of your data with the response and all p predictors
\end{ldescription}
\end{Arguments}
%
\begin{Details}\relax
Finds all p choose 2 combinations of predicotrs in the solution set
\end{Details}
%
\begin{Value}
A vector of the RSS values for each candidate predictor
\end{Value}
%
\begin{Author}\relax
Kirk Gosik
\end{Author}
%
\begin{SeeAlso}\relax
\code{model.matrix}
\end{SeeAlso}
\inputencoding{utf8}
\HeaderA{strong\_order3}{Creating interactions based off of strong heredity principle}{strong.Rul.order3}
%
\begin{Description}\relax
Helper function to give all possible order-3 interactions following the strong
heredity principle.
\end{Description}
%
\begin{Usage}
\begin{verbatim}
strong_order3(S, data)
\end{verbatim}
\end{Usage}
%
\begin{Arguments}
\begin{ldescription}
\item[\code{S}] vector of solution predictor selected from previous steps of the procedure

\item[\code{data}] data.frame of your data with the response and all p predictors
\end{ldescription}
\end{Arguments}
%
\begin{Details}\relax
Finds all p choose 3 combinations between the predicotrs in the solution set.
\end{Details}
%
\begin{Value}
A vector of the RSS values for each candidate predictor
\end{Value}
%
\begin{Author}\relax
Kirk Gosik
\end{Author}
\inputencoding{utf8}
\HeaderA{weak\_order2}{Creating interactions based off of weak heredity principle}{weak.Rul.order2}
%
\begin{Description}\relax
Helper function to give all possible order-2 interactions following the weak
heredity principle.
\end{Description}
%
\begin{Usage}
\begin{verbatim}
weak_order2(S, C, data)
\end{verbatim}
\end{Usage}
%
\begin{Arguments}
\begin{ldescription}
\item[\code{S}] vector of solution predictor selected from previous steps of the procedure

\item[\code{C}] vector of candidate predictors to consider in this step of the procedure

\item[\code{data}] data.frame of your data with the response and all p predictors
\end{ldescription}
\end{Arguments}
%
\begin{Details}\relax
Finds all p choose 3 combinations between the predicotrs in the solution set and the
predictors in the candidate set.
\end{Details}
%
\begin{Value}
A vector of the RSS values for each candidate predictor
\end{Value}
%
\begin{Author}\relax
Kirk Gosik
\end{Author}
\inputencoding{utf8}
\HeaderA{weak\_order3}{Creating interactions based off of weak heredity principle}{weak.Rul.order3}
%
\begin{Description}\relax
Helper function to give all possible order-3 interactions following the strong
heredity principle.
\end{Description}
%
\begin{Usage}
\begin{verbatim}
weak_order3(S, C, data)
\end{verbatim}
\end{Usage}
%
\begin{Arguments}
\begin{ldescription}
\item[\code{S}] vector of solution predictor selected from previous steps of the procedure

\item[\code{C}] vector of candidate predictors to consider in this step of the procedure

\item[\code{data}] data.frame of your data with the response and all p predictors
\end{ldescription}
\end{Arguments}
%
\begin{Details}\relax
Finds all p choose 3 combinations between the predicotrs in the solution set and the
predictors in the candidate set.
\end{Details}
%
\begin{Value}
A vector of the RSS values for each candidate predictor
\end{Value}
%
\begin{Author}\relax
Kirk Gosik
\end{Author}
\printindex{}
\end{document}
